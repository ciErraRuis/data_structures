\documentclass[UTF8]{ctexart}
\usepackage{geometry, CJKutf8}
\geometry{margin=1.5cm, vmargin={0pt,1cm}}
\setlength{\topmargin}{-1cm}
\setlength{\paperheight}{29.7cm}
\setlength{\textheight}{25.3cm}

% useful packages.
\usepackage{amsfonts}
\usepackage{amsmath}
\usepackage{amssymb}
\usepackage{amsthm}
\usepackage{enumerate}
\usepackage{graphicx}
\usepackage{multicol}
\usepackage{fancyhdr}
\usepackage{layout}
\usepackage{listings}
\usepackage{float, caption}

\lstset{
    basicstyle=\ttfamily, basewidth=0.5em
}

% some common command
\newcommand{\dif}{\mathrm{d}}
\newcommand{\avg}[1]{\left\langle #1 \right\rangle}
\newcommand{\difFrac}[2]{\frac{\dif #1}{\dif #2}}
\newcommand{\pdfFrac}[2]{\frac{\partial #1}{\partial #2}}
\newcommand{\OFL}{\mathrm{OFL}}
\newcommand{\UFL}{\mathrm{UFL}}
\newcommand{\fl}{\mathrm{fl}}
\newcommand{\op}{\odot}
\newcommand{\Eabs}{E_{\mathrm{abs}}}
\newcommand{\Erel}{E_{\mathrm{rel}}}

\begin{document}

\pagestyle{fancy}
\fancyhead{}
\lhead{颜瑞, 3210102064}
\chead{数据结构与算法第四次作业}
\rhead{Oct.21st, 2024}

\section{测试程序的设计思路}

首先本次测试我们需要测试以下功能的正确性:

1. constructor, copy constructor, move constructor, deconstructor

2. copy assignment, move assignment

3. 前后插入,删除

4. front, back

5. ++ / --

6. 解引用(*)

7. == / !=

8. 不同模板类

\vspace{\baselineskip}


测试方法分别如下:

1. constructor: 初始化一个初始值为1,2的链表b,打印b看是否符合预期。

2. copy constructor: 通过拷贝b构造一个链表c。打印c看是否符合预期。再修改b的front和back后再次打印c。若c无变化则正确。

3. copy assignment: 先初始化一个链表a,再将b拷贝赋值给a。打印a和b。再修改b,若a无变化则正确。

4. move constructor: 通过a移动构造一个链表d。打印d的值,并查看a是否为empty。若a的值成功赋给了e且a为空则正确。

5. move assignment: 将构造换为赋值后同上。

6. deconstructor: 用valgrind检查是否存在内存泄漏。

7. push\_front/back, pop\_front/back: 在修改b时使用这四个函数并检查结果是否符合预期。

8. front, back: 用front, back输出push\_front/back的结果。

9. *, ++, --: 将b.begin()赋值给一个iterator it,分别输出$*it++$, $*++it$, $*--it$, $*it--$,并在每次执行完后输出*it。观察是否符合预期。

10. 模板类: 初始化一个模板类为std::string的链表f,push\_front插入"runis", push\_back插入"cierra", 打印f若输出"cierra runis"则正确。

11. != / ==: 由于!=等于==取反,只需测试==的正确性。在3.中将b拷贝赋值给a后,输出$a.begin() == b.begin()$。由于iterator的==输出的是是否指向同一node,所以虽然此时二者node的值相等,结果仍应该为0。
   接着测试$b.begin() == b.begin()$应该输出1。最后,给一个iterator it1 赋值为b.begin(),将b.begin()修改后测试$it1 == b.begin()$,应该为1。

\section{测试的结果}

   测试结果一切正常。
   
   我用 valgrind 进行测试,发现没有发生内存泄露。

\end{document}

%%% Local Variables: 
%%% mode: latex
%%% TeX-master: t
%%% End: 
